\textbf{\centering \large Abstract \\}
The increasing penetration of distributed energy resources (DERs)—particularly solar photovoltaics and battery energy storage—presents significant opportunities for optimizing distribution system operations. However, coordinating these resources over extended time horizons through multi-period optimal power flow (MPOPF) becomes computationally prohibitive as network size and planning duration increase. The coupling of spatial variables (network topology) and temporal variables (battery state-of-charge dynamics) creates problems with tens or hundreds of thousands of decision variables, making scalable solution strategies essential for practical implementation.

This research investigates decomposition-based approaches to address MPOPF computational challenges. We first establish fundamental modeling trade-offs by comparing linear and nonlinear power flow formulations on the IEEE 123-bus system, revealing when simplified models suffice versus when accurate nonlinear representations become necessary. This analysis informs the choice of decomposition strategies for different operating conditions and accuracy requirements.

Spatial decomposition methods such as the Equivalent Network Approximation (ENApp) approach prove highly effective for partitioning large distribution networks and accelerating computation. However, while valuable for large systems, spatial decomposition alone provides limited relief for long-horizon MPOPF problems due to persistent temporal coupling through battery dynamics. This limitation motivates the development of temporal decomposition algorithms.

We develop and evaluate Differential Dynamic Programming (DDP) for temporal decomposition of MPOPF. DDP exploits the sequential structure of battery state-of-charge evolution to decompose multi-period problems into tractable subproblems solved through backward-forward iterations. Results on IEEE 123-bus variants demonstrate that DDP achieves good solution quality and rapid initial convergence. However, our findings also reveal that further research is needed to ensure algorithmic robustness and numerical stability, particularly for larger networks and extended horizons.

As a complementary approach, we investigate temporal ADMM (tADMM), which enables parallel optimization across time periods through consensus-based coordination. Initial results show strong convergence properties, offering an alternative temporal decomposition strategy with different computational trade-offs compared to DDP's sequential structure.

The overarching objective of this thesis is to develop scalable temporal decomposition algorithms for MPOPF that enable day-ahead operational planning (T = 24 to 96 time periods) in realistic distribution networks. While the ultimate goal targets large three-phase systems, this work establishes the foundation using single-phase medium-scale networks such as IEEE 123-bus, demonstrating feasibility before scaling to more complex configurations. By advancing temporal decomposition methods, this research enables efficient coordination of DERs in future distribution systems with high renewable penetration and extensive energy storage deployment.
