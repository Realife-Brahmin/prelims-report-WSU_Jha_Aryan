\textbf{\centering \large Abstract \\}
The socioeconomic losses from the recent unprecedented incidents in the electric power systems suggest the need for alternative planning strategies that account for the expected and extreme events that are less likely to occur. Such high impact low probability (HILP), or black swan, events are typically weather-related, have accounted for billions of dollars in economic losses, and left customers in the dark for several days. Furthermore, the proliferation of distributed energy resources (DERs) on the distribution grid indicates that system operators can also plan resilience from the customer's end, forming intentional microgrid islands when needed. However, existing planning strategies only minimize the expected operating cost and do not explicitly include the risk of extreme events. With the increasing frequency of black swan events in the current scenario, system operators should focus on the HILP events and find the optimal trade-off decision to maximize resilience with available resources. This proposal aims to investigate the impact of extreme weather events, hurricanes, and floods, on the power grid and propose planning solutions to enhance the grid's resilience. Firstly, we propose a modeling framework to assess the spatiotemporal compounding effect of hurricanes and storm surges on electric power systems. The spatiotemporal probabilistic loss metric helps system operators identify the potential impact and vulnerable components as the storm approaches. Secondly, we develop a risk-averse two-stage stochastic optimization framework for resilience planning of power distribution systems against extreme weather events. The resource planning strategy involves minimizing a risk metric, conditional value-at-risk (CVaR) while adhering to budget constraints for planning. The main idea is to identify a trade-off between risk-averse and risk-neutral planning solutions to maximize the energization of critical loads when a HILP event is realized. The problem is also extended to determine the trade-off among different resources when system operators have a limited budget. This facilitates the selection of specific resources from a portfolio of various resources that can optimally restore critical loads during the realization of a HILP event. In the future, the work will be extended to a larger and broader landscape with an analysis based on realistic extreme weather events and their impact on the power system. The goal will be to create a generic simulation platform capable of generating and assessing the impact of such events on the electric power grid. The work will also include large-scale integrated operational solutions to enhance the resilience of future power grids. Furthermore, advanced parallel algorithms based on dual decomposition methods will be explored for scalable implementation of stochastic programming problems.
