\section{Motivation}
Extreme weather events can have a devastating impact on the electric power grid, posing serious concerns for personal safety and national security. In the US alone, extreme weather events between 2003 and 2012 caused nearly 700 power outages, with 80-90\% of these outages resulting from power distribution system failures~\cite{DOE_res}. In 2021, there were 20 natural disasters in the US that caused losses exceeding \$1 billion each, highlighting the urgent need for more resilient power grid infrastructure~\cite{noaa}. Hurricanes, in particular, account for over a trillion dollars in economic losses and are responsible for a significant number of power outages in the US~\cite{cost_NOAA, 2020}. During landfall, as the strong hurricane wind field traverses inland, it propels a huge water body known as a storm surge flooding the coastal regions. The storm surge is sometimes the most destructive part of a hurricane and accounts for considerable damage~\cite{surge_web}. For instance, Hurricane Ida caused about \$55 billion in damages in Louisiana alone due to wind and storm surge damage, with additional flooding damage of about \$23 billion in the Northeastern US~\cite{IDA_NOAA}. Almost 1.2 million customers experienced power outages across eight different states. Hurricane Ian recently had a devastating impact in Florida and is expected to have incurred billions of dollars in losses, with a peak of about 2.7 million customers in a power outage~\cite{DOE_IAN}. The frequency of such high-impact, low-probability (HILP) events has increased at an alarming rate, costing about \$152.6 billion in climate-related disasters in 2021 alone in the US. Therefore, there is an urgent need to identify the potential impacts of these hazards on electric power systems. 

The resulting socio-economic losses and the power grid's vulnerability to extreme weather events necessitate the incorporation of resilience in system planning to account for not only the expected events but also the extreme events that are less likely to occur. Towards this goal, different utilities have spent millions of dollars deploying smart grid technologies such as distribution automation with automated feeder switching, intentional islanding (microgrid), and upgrading vulnerable feeders and substations~\cite{9120304}. However, with the increasing frequency and severity of weather-related events, a more systematic approach to smart grid expenditures is required to identify appropriate system upgrade solutions for strengthening system resilience. This proposal aims to build a risk-averse planning framework that can enhance the resilience of power distribution systems against extreme weather events.

In recent years, there have been several advancements in weather prediction models. Still, they have not been adequately utilized to analyze the potential impact of upcoming natural hazards on the power grid. Such predictive information is essential to power grid planners and operators to reduce the effects when an event is realized~\cite{9942328, 9810633}. For instance, system operators can identify potential substations that could be inundated due to storm surges and proactively disconnect them to avoid equipment damage and facilitate fast restoration. These long-term planning strategies can help planners identify vulnerable transmission lines and propose line-hardening strategies. This work aims to identify the spatiotemporal impact assessment of hurricanes and storm surges on electric power systems, providing a weather-grid impact model for outage risk assessment that accurately models the time-varying impact of extreme weather events on critical infrastructure such as power systems.
