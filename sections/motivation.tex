\clearpage
\section{Motivation}

Optimal power flow (OPF) methods have become vital for efficiently managing distributed energy resources (DERs) such as photovoltaic (PV) and battery energy storage systems (BESS) at the grid edge, aiming to enhance system-level objectives including reliability, resilience, and cost-effectiveness \cite{Jha2024, Nazir2019Jun}. BESS, by mitigating the fluctuations of intermittent DERs, transforms the OPF problem from a classic single-period formulation into a multi-period, time-coupled optimization that demands intricate modeling and advanced computational methods \cite{Jha2024, ddp_sugar_01}.

Centralized OPF (COPF) approaches typically rely on non-convex formulations—often based on the nonlinear branch flow model \cite{bfm01}—to coordinate controllable devices. Although such methods can provide accurate solutions for small networks \cite{Gabash, Jha2024, ddp_sugar_01}, computational time and scalability issues limit their utility in large-scale, real-world systems. Metaheuristic and evolutionary algorithms \cite{Nazir2018Jun} offer alternative solutions but generally struggle with local optimality and slow convergence, especially for high-dimensional multi-period OPF tasks.

To improve tractability, convex relaxations and linear approximations such as LinDistFlow \cite{Gan} have been widely employed \cite{Jha2024, Gan, ddp_sugar_01}. These methods deliver fast convergence but introduce non-negligible optimality gaps, particularly as network size and DER/BESS penetration increase, which has not been fully quantified in existing research \cite{Jha2024}. Recent studies have highlighted this gap, advocating for a systematic comparison of non-convex and linearized frameworks across varied network conditions \cite{Jha2024, Gan}.

Recognizing the limitations of centralized and linear-programming-based solutions, recent research—including our own—has focused on spatial and temporal decomposition for scalable multi-period OPF. Our group adapted the Equivalent Network Approximation (ENApp) framework to enable distributed MPOPF (MPDOPF) in battery-integrated networks, leveraging spatial partitioning to allow parallel solution of local OPF subproblems and reduce global computational burdens \cite{Jha2025}. This approach harnesses the radial structure of distribution networks and significantly accelerates convergence while maintaining solution fidelity. Moreover, our models introduce a battery loss term exclusively in the objective function, preventing simultaneous charge/discharge operations and preserving problem convexity without integer variables \cite{Jha2025}.

Extensive validation against industry-standard IEEE test systems demonstrates that our ENApp-based distributed MPOPF delivers superior scalability and optimality compared with conventional centralized frameworks. These studies also provide a robust, side-by-side comparison of LinDistFlow and branch flow-based nonlinear models for MPOPF across real distribution network scenarios \cite{Jha2024}.

Major remaining research gaps include the scalability and real-time applicability of centralized MPOPF approaches \cite{Gabash, ddp_sugar_01, mpopf2021, mpopfstorage2021, largevopfw2021}, and the slow convergence and master controller dependencies in existing distributed frameworks such as Benders decomposition \cite{Wu}. The contribution of this work is the development and benchmarking of a spatially distributed multi-period OPF algorithm, overcoming these challenges and enabling practical, accurate, large-scale coordinated grid-edge resource management.

