\clearpage
\section{Objectives and Task Lists}
The overarching goal of this work is to develop a computationally scalable framework for solving the Multi-Period Optimal Power Flow (MPOPF) problem in active distribution systems. The proposed framework aims to address the inherent complexity and non-convexity of MPOPF formulations -- arising from temporal coupling of decision variables and high-dimensional system models -- by designing and implementing decomposition-based algorithms. The resulting framework is expected to enable tractable and near-optimal solutions for realistic time horizons and large distribution feeders, while preserving the physical fidelity of the underlying models. The following tasks have been completed or are currently being pursued toward achieving the objectives of this study:

\begin{enumerate}
    \item \textbf{Literature Review on Multi-Period Optimal Power Flow:} A comprehensive review was conducted on existing formulations of the MPOPF problem for active distribution systems. This provided insights into modeling temporal dependencies, handling storage dynamics, and integrating distributed energy resources in multi-period settings.

    \item \textbf{Study of Decomposition-Based Optimization Techniques:} A review of decomposition algorithms, including Bilevel optimization methods, ADMM, and Differential Dynamic Programming (DDP), was carried out to understand their suitability for large-scale and temporally coupled optimization problems. The study highlighted key convergence properties and trade-offs between spatial and temporal decompositions.

    \item \textbf{Implementation of Spatial Decomposition for MPOPF:} A spatially decomposed MPOPF formulation was implemented and tested on benchmark distribution systems. The method demonstrated improved scalability compared to monolithic optimization, though it remained limited in addressing the temporal coupling present in long-horizon studies.

    \item \textbf{Development and Evaluation of Differential Dynamic Programming (DDP):} The DDP algorithm was formulated and implemented for the MPOPF problem. While the approach yielded solutions close to those obtained from brute-force optimization, it exhibited oscillatory convergence behavior and lacked strong theoretical guarantees. Further research is required to enhance its stability and convergence properties.

    \item \textbf{Implementation of Temporal ADMM for MPOPF:} A temporal Alternating Direction Method of Multipliers (ADMM) approach is being developed to address the scalability challenges associated with increasing time horizons. Preliminary results on a copper-plate system show excellent convergence, and current efforts focus on extending the implementation to LinDistFlow-based distribution models.

    \item \textbf{Validation and Scalability Testing:} The final phase will involve validating the proposed algorithms on a large-scale system like the 9500-node (three-phase) feeder. The goal is to demonstrate the framework’s scalability for realistic horizons (e.g., $T=96$ i.e. $1$ day at $15$-minute intervals) and its applicability to operational optimization in distribution grids.

    \item \textbf{Exploratory Study on Multiple-Source Optimal Power Flow (MS-OPF):} As a future research direction, the framework will be extended to accommodate multi-source configurations, enabling coordinated optimization across multiple substations and zones in distribution networks.
\end{enumerate}
