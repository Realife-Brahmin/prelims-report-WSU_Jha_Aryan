\section{Objectives and Task Lists}
The major goal of this work is to create a framework for multi-resource planning that improves the resilience of electric power systems. The proposed framework starts with a risk-averse active distribution system planning approach, which formulates the distributed siting and sizing problem using a two-stage stochastic programming method. Additionally, the framework includes a risk-based probabilistic resilience metric that uses a multi-criteria decision-making process to evaluate attribute-based and performance-based parameters in the distribution system. The next step involves assessing the spatiotemporal impact of extreme weather events on the bulk grid level to aid long-term investment planning. The following tasks have been completed to achieve the objectives of this study: 

\begin{enumerate}
    \item Literature review on Existing Resilience-based Planning: A detailed literature review is conducted on existing planning methods for power distribution system. The literature review provided an insight on how the planning objectives were formulated and how the planning-based simulation framework should be developed.  
    
    \item Review of stochastic optimization modeling techniques and algorithms: The literature review on stochastic optimization modeling techniques provided an insight on how two-stage stochastic optimization technique works and what are the important challenges in solving such problem. Along with the modeling techniques, the literature review on this domain provided some insights on the algorithms that can be leveraged in alleviating the computational complexity in solving the problems.  
    
    \item Multi-criteria Decision Making Process to Quantify Resilience: Highlighted the notion that the resilience is dependent on multiple system criteria which should be included to quantify the system's resilience. Risk-based resilience metric based on multiple criteria in a system is introduced to incorporate the impact of stochastic HILP event. 
    
    \item Risk-based Active Distribution System Planning: A risk-based distribution system planning framework is developed based on two-stage stochastic programming method. The operational stage has several restoration features such as DG-assisted intentional islanding and automatic tie-switches while the planning resources include an additional line hardening feature.  
    
    \item Literature review on an impact assessment extreme weather event at the bulk grid level: Existing methods are studied on how extreme weather event impacts are studied at the bulk grid level. The study is also conducted on suitable statistical methods to generate hurricanes and flood scenarios for system-level study. 
    
    \item Spatiotemporal Impact Assessment of Hurricanes and Storm Surges on Power Systems: A simulator is developed that can generate spatiotemporal hurricane scenarios and analyze the storm surge based on the generated hurricane scenarios.   
    
\end{enumerate}
